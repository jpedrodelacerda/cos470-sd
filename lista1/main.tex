\documentclass{article}
\usepackage[utf8]{inputenc}
\usepackage[utf8]{inputenc}
\usepackage{moderncvcompatibility}

\sloppy
\usepackage{graphicx,url}

\title{COS470 - Sistemas Distribuídos: \\
  \large Lista 1}
\author{João Lacerda\\ \email{jpedrodelacerda@poli.ufrj.br}}
\date{April 2022}

\begin{document}

\maketitle

\section{Sistemas de Computação}

Os Sistemas de Computação são utilizados em problemas de computação em alto desempenho (\textit{high-performance computing}).
O objetivo deste tipo de solução é proporcionar maior escalabilidade e pode ser divido em dois grandes subgrupos: \textit{Cluster} e \textit{Grid}.

\subsection{\textit{Cluster}}

Quando falamos de computação com \textit{clusters}, os nós do sistema costumam ser homogêneos, isto é, possuem \textit{hardware} semelhantes e rodam o mesmo Sistema Operacional.\\
Outra característica importante é que os aglomerados estão conectados via LAN (\textit{Local-Area Network}). Dessa forma, o sistema consegue paralelizar o processamento de dados sem aumentar de maneira considerável a latência/desempenho.

\subsection{\textit{Grid}}

Neste subgrupo, a disposição dos nós é diferente: costumam ser federações de sistemas de computadores que podem ser mais heterogêneos e possuem diferentes domínios administrativos.\\
Dessa forma, organizações virtuais são criadas visando permitir a colaboração de diferentes instituições. Essas organizações compartilham recursos internamente entre seus processos, que podem variar de servidores de computação, dispositivos de armazenamento e até bancos de dados.


\section{Sistemas de Informação}

Os Sistemas de Informação são os que mais usamos no dia-a-dia: sendo sua forma mais comum servidores e clientes se comunicando.\\
Nessa comunicação, são trocadas informações para que uma determinada operação seja realizada.\\
O nível de integração pode variar, sendo possível múltiplas requisições (podem ser servidores diferentes) serem enviadas com uma única finalidade, ou seja, para uma única operação e serem executadas de forma transacional: ou a transação acontece como um todo ou a operação é abortada.
As transações devem aderir aos princípios \textbf{ACID} (\textbf{A}tômica, \textbf{C}onsistente, \textbf{I}solada e \textbf{D}urável) para manter a integridade dos dados.\\
Por ser composto por nós heterogêneos, uma vantagem dos Sistemas de Informação é a sua capacidade de adaptação: as partes do sistema podem ser alocadas em \textit{hardware} específicos para aumentar a performance, isto é, partes que cuidam de operações \textit{CPU-bound} serão alocadas em máquinas com CPUs mais rápidas enquanto partes \textit{memory-bound} podem ter mais memória disponível. Além disso, a segmentação dos diferentes tipos de operações também permite uma maior escalabilidade do sistema: em picos de tráfegos, por exemplo, apenas as partes que realmente estão sendo utilizadas podem ser escaladas, melhorando assim, a utilização dos recursos e a performance geral do sistema.

\section{Sistemas Pervasivos}

Os Sistemas Pervasivos são, intrínsicamente, distribuídos. Em sua maioria, compostos por dispositivos simples e de baixo custo, mas em grande quantidade.\\
O crescimento dos Sistemas Pervasivos se dá motivos: ele compreende justamente os dispositivo que encontramos em nosso cotidiano e com o avanço do paradigma da Internet das Coisas, a integração dos dispositivos se torna cada vez mais comum, aumentando ainda mais o escopo deste tipo de sistema.\\
Uma característica interessante dos Sistemas Pervasivos é que a comunicação entre eles é sumariamente sem fio.

\end{document}
